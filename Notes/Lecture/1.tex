\documentclass[class=article,crop=false]{standalone}
\usepackage[subpreambles=true]{standalone}
\begin{document}
\lecture
\subsection{Syllabus}
\subsubsection{Why Modern Physics?}
Already learned about mechanics, electromagnetism, and thermodynamics, which forms the basis of "classical physics", however, classical physics fails to explain many observed phenomena, for example, black-body radiation, stability of atoms/radioactivity. \\\\
\textbf{Modern Physics} refers to developments that begun ~1900, leading to relativity and quantum theories and their applications. Is not irrelevant - many very important applications i.e. GPS, semiconductors, medical imaging, lasers. \\

\subsubsection{Course Logistics}
\begin{description}
	\item [$\bullet$] Office hours Mon/Wed @ 1 - 2:30 pm
	\item [$\bullet$] Krane's Modern Physics 3rd or 4th edition, electronic versions are fine
	\item [$\bullet$] Weekly reading assignments posted on Canvas
	\item [$\bullet$] Homework posted on Canvas typically on Wednesdays, due the following Thursday at start of class (over canvas)
	\item [$\bullet$] Late assignments are not accepted, but lowest HW score will be dropped
	\item [$\bullet$] Professor is teaching from CERN. You can go to the classroom but the teacher won't be there.
\end{description}

There are no stupid questions, please ask questions. Being able to ask questions is an important skill. \\
Questions outside of class can be answered during office hours or on Canvas discussion boards. \\
Physics workshop is good for homework help, and there is also tutoring.\\

There will be at least one TA, assigned at some point. \\

\subsection{Introduction to Special Relativity}
Relativity  - How are events observed in different reference frames? \\
Nothing new -- Galileo \& Newton knew the Relativity Principle applied to mechanics, that the laws of mechanics are the same in all inertial reference frames. \\
Inertial reference frame? \\
Reference frame in which newton's 1st law is valid, i.e. a non-accelerating frame.\\
Newton's laws are the same as seen in different inertial reference frames (invariant) \\

\subsubsection{Galilean Transformations}
Consider two different reference frames S and S', where S' is moving with constant uniform velocity $u$. \\
Assume that time is the same for all observers (t = t') \\

If there is an Object P at coordinates x', y', z', the Observer O would observe object at x = x' + ut, y = y', z = z'. \\
\newpage
\begin{result}[Galilean Transformations]
	Coordinate Transformation:
	\[x' = x - ut, y' = y, z' = z\]

	Velocity Transformation: 
	$$ \frac{d}{dt}(x') = \frac{dx}{dt} - u \rightarrow V'_x = V_x - u $$
	$$ V'_y = V_y $$
	$$ V'_z = V_z $$

	Acceleration Transformation:
	$$ \frac{d}{dt}(V'_x) = \frac{dV_x}{dt} \implies a'_x = a_x $$
	$$ a'_y = a_y $$
	$$ a'_z = a_z $$
\end{result}

Newton's second law holds for all observers (it obeys the relativity principle), and that no inertial reference frame is special. \\

\subsubsection{The need for the ether}
What about Maxwell's equations of electromagnetism? \\
Electromagnetic waves (light) propogate with a velocity $ c = 3 \times 10^8 m/s $. \\

Instead of thinking of P as a point, think of it as a light source. Would $V_x = V'x + u$ imply $ V_x = c + u $ ?? \\

Maxwell's equations are not invariant under Galilean transformation (evident from how Maxwell's equations predicted speed of light!) \\
Speed of something depends on which interial frame you are observing it from - are Maxwell's equations only correct in a certain frame? \\

\subsubsection{Michelson-Moerley experiment}
Physicists in 19th century postulated that light traveled trough a medium (ether). \\
Speed of light predicted by Maxwell's equations was interpreted as that relative to the ether, i.e. ether preposed as an absolute reference system from which c was constant. \\
This would mean that the laws of physics is not different, because you're simply moving relative to the ether! \\

Michelson and Morley used an interferometer mounted on a 1.5m stone slab floating in mercury, to reduce the effect of vibrations and isolate the setup. \\

\subsubsection{Michelson interferometer}
Monochromatic beam of light split in two, travel different paths, then recombined \\
Phase difference between combined beams cause fringes to appear \\

\textbf{Two contributions} \\
\begin{enumerate}
	\item Phase difference AB-AC, one beam might travel longer distance \\
	\item Time difference when moving parallel to ether direction vs perpendicular to it - - rotating by 90 degrees should isolate this! \\
\end{enumerate}

Expectation: Fringe shift as interferometer was rotated with respect to the ether velocity. Michelson-Moerley could resolv 0.005 periods and expected 0.4 periods of shift. 

\textbf{This was not observed} \\

OK, so maybe it was the rotation of the earth around the sun that canceled this out. So they repeated half a year later, and still a null result. Speed of light is always c.

\subsubsection{Modern-day interferometers}
LIGO: Laser Interferometer Gravitational-Wave Observatory \\
September 14, 2015: FIrst direct detection of gravitational waves and the first observation of a binary black hole merger. \\
Measuring $\frac{\Delta L}{L}$ on the order of $10^{-21}$, based on the difference betwen the two sites. \\
This is a direct development of the interferometer Michelson and Morley used.

\begin{result}[Einstien's Postulates]
	$ \bullet $ \textbf{The principle of relativity:} The laws of physics are the same in all intertial reference frames. There is no way to detect absolute motion, and no preferred inertial system exists. \\
	$ \bullet $ \textbf{The constancy of the speed of light:} The speed of light in free space has the same value c in all inertial reference frames. \\
	$ V'_x = V_x + u $ is \emph{\textbf{not}} true.
\end{result}

\end{document}
