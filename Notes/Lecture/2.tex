\documentclass[class=article,crop=false]{standalone}
\usepackage[subpreambles=true]{standalone}
\begin{document}
\newpage
\lecture
\subsection{Relativity of Time}
\begin{question}[Consequence of Posulates]
	What is the time interval measured by a person on a vehicle? \\
	Person in vehicle sees $\Delta t = \frac{2d}{c} $ \\
	Person outside vehicle sees the light travel along a triangle. \\
	total x distance : $v \Delta t$ \\
	total light distance : $c \frac{\Delta t}{2}$ \\
	Pythagorean theorem gives : $d^2 + (\frac{v \Delta t}{2})^2 = (\frac{c \Delta t}{2})^2 $ \\
	$$ \Delta t = \frac{2d}{\sqrt{c^2 + v^2}} $$
	$$ \Delta t = \frac{\Delta t_0}{\sqrt{1 - \frac{v^2}{c^2}}} $$
	\begin{result}[Gamma Factor]
		$$ \gamma = \frac{1}{\sqrt{1 - \frac{v^2}{c^2}}} $$ \\
		$$ \gamma > 1\ for\ v < c $$
	\end{result}
$$ \Delta t = \gamma \Delta t_0 $$
	... Time dilation!
\end{question}

\subsubsection{Cosmic Muons}
Lifetime of muons is ~2.2 $\mu$ s \\
Produced in the atmosphere (atmosphere ~100km high) \\
Average max distance:
$$ L = 2.2 \cdot 10^{-6}\ s \cdot 3 \cdot 10^8\ m/s = 660\ m $$
Time dilation is a real effect, even if not noticeable at normal speeds! \\

$\mu$ lifetime according to observer on Earth: \\
$$ \tau = \gamma \tao_0 $$
$$ \tau_0 = 2.2 \mu s $$
$$ if\ v = 0.99999c,\ \gamma \approx 224 $$
$\tau \approx 490 \mu s!$
Length travelled:
$$ L = \tau \cdot c = 490 \cdot 10^{-6} \cdot 3 \cdot 10^8 = 147,000\ m $$
\begin{question}[Two stars]
	We have an observer on earth, and two stars in space. There is a spaceship travelling from star 1 to star 2 at velocity v. \\
	\begin{answer}[The observer at rest on Earth w.r.t the two stars, measures:]
		Distance between stars : $L_0$ \\
		Time for spaceship to travel : $\Delta t = \frac{L_0}{v}$\\
	\end{answer}
	\begin{answer}[Observer on spaceship traveling with speed v from star to star, measures:]
		Smaller time of tavel : $$\Delta t_0 = \frac{\Delta t}{\gamma}$$ \\
		Distance measure : $$L = v \Delta t_0 = \frac{v \Delta t}{\gamma} $$ \\
		$$ L = \frac{v \Delta t}{\gamma} = \frac{L_0}{\gamma} $$
		.. Length contraction!
	\end{answer}
\end{question}
Length contraction takes place in direction of motion. Objects do not "shrink" - there is just a difference in measured length by different observers.
\subsection{Relativistic Velocty Addition}
$$\frac{v + u}{1 + \frac{vu}{c^2}}$$

\subsection{Relatavistic Doppler Shift}
Classical Doppler effect for sound waves, $f' = f * \frac{v + v_0}{v - v_s}$\\

For light, with no medium, and no preferred reference frame...

\end{document}

