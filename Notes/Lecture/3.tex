\documentclass[class=article,crop=false]{standalone}
\usepackage[subpreambles=true]{standalone}
\begin{document}
\lecture
\newpage
\subsection{Doppler Shift}
\begin{result}[Classical Doppler Shift]
	v = speed of waves in medium \\
	$v_0$ = speed of observer w.r.t. medium \\
	$v_s$ = speed of source w.r.t. medium \\
	$$ f' = f \cdot \frac{v + v_0}{v-v_s}$$
\end{result}

\begin{result}[Relativistic Doppler Shift]
	f = frequency in source rest frame \\
	f' = frequency measured by some observer moving away from source w/ speed u.
	$$ f = \frac{N}{\Delta t_0} $$
	$$ f' = \frac{c}{\lambda '} $$
	$$ \lambda ' = \frac{u \Delta t' + c}{N} $$
	$$ f' = \frac{cN}{u \Delta t' + c \Delta t'} $$
	$$ f' = \frac{cf \Delta t_0}{(u + c) \Delta t'} $$
	$$ f' = f \sqrt{\frac{1 - u/c}{1 + u/c}} $$ 
	No dependence on whether the source is moving or the observer is moving.\\
	Note: If O' is moving toward the source, u is replaced with -u.
\end{result}

\begin{question}
	Distant galaxy moving away from earth. Blue hydrogen line expected at wavelength 434 nm measured at 600 nm. What is the speed of the galaxy relative to earth?
\end{question}
\begin{answer}
	$$ f = c/\lambda $$
	$$ c/\lambda ' = c/\lambda \sqrt{\frac{1 - u/c}{1 + u/c}} $$
	$$ (\lambda / \lambda ')^2 = \frac{1-u/c}{1+u/c} $$
	$$ u = c \frac{1 - (\frac{\lambda}{\lambda '})^2}{1 + (\frac{\lambda}{\lambda '})^2} = .31 c $$
\end{answer}

\subsection{Lorentz transformations}
Galilean coordinate transformations are not consistent with Einstien's postulates, and don't agree with experiments at high speeds. \\
New coordinate transformation must fulfill... linearity in space/time, reduce for Galilean version for $v << c$, consistent with Einstien's postulates. \\

$$ x' = \frac{x - ut}{\sqrt{1-u^2/c^2}} $$
$$ y' = y,\ z' = z, $$ 
$$ t' = \frac{t- u/c^2 x}{\sqrt{1-u^2/c^2}} $$

Lorentz transformation is just a generalized way to write what happens w/in a coordinate system, and the other rules can follow from these. \\

\subsubsection{Velocity transformation}
We can derive velocity transformation by differeniating the Lorentz coordinate transformation (space \& time!) \\
\begin{result}
	$$ V'_x = \frac{V_x - u}{1-V_x u/c^2} $$
	$$ V'_y = \frac{V_y \sqrt{1-u^2/c^2}}{1-V_y u/c^2} $$
	$$ V'_z = \frac{V_z \sqrt{1-u^2/c^2}}{1-V_z u/c^2} $$

Velocity transformation in three dimensions. If we assume movement in the x direction only, you still have this complicated stuff in other dimensions because you're also deriving with respect to time.
\end{result}

\subsubsection{What would the photon see?}
Infinitely length contracted \& if it had a clock, that clock would not be moving. BUT it is not an inertial reference frame in which measurements mean anything. Mathematical singularity -- not defined -- can't divide by zero!

\subsubsection{Simultaneity}
Time acording to O for light to reach two clocks the same distance from a light source:
$$ t = \frac{L}{2c} \implies\ clock\ 1\ \&\ 2\ are\ syncronized $$
From point of view of moving observer O'? Use Lorentz transformations!\\
Clock 1 receives signal at : \\
$$ t'_1 = \frac{t_1 - (u/c^2)x_1}{\sqrt{1-u^2/c^2}} = \frac{L/2c}{\sqrt{1-u^2/c^2}} $$
clock 2 receives signal at : \\
$$ t'_2 = \frac{t_2 - (u/c^2)x_2}{\sqrt{1-u^2/c^2}} = \frac{L/2c - (u/c^2)L}{\sqrt{1-u^2/c^2}} $$
Difference:
$$ \Delta t' = t'_1 - t'_2 = \frac{uL/c^2}{\sqrt{1-u^2/c^2}} $$

\subsection{Relatavistic Kinematics}
\subsubsection{Relativistic Momentum}
Consider an elastic collision as observed in the rest frame of observer O', where a particle of mass 2m, moving to the left at -.750c, collides with a particle of mass 2m at rest with respect to O'. \\
\begin{question}[Classical Interpretation]
	$$p_i' = m_i v_{1i}' + m_2' v_{2i}' $$
	$$ = 2m \cdot 0 + m(-0.750c) $$
	$$ = -.75mc $$
	Final:
	$$ p_f' = m_1' v_{if}' + m_2' v_{2f}' $$
	$$ = 2m v_{1f}' + m v_{2f}' $$
	$$ p_i' = p_f' \rightarrow v_{1f}' = -0.5c, v_{2f}' = 0.25c $$
	Momentum is conserved. \\
	\bigskip
	Now assume that refernce frame of O' is moving with velocity 0.550c in the positive x direction relative to observer O. What does O think about momentum conservation? \\
	$$ v_{1i} = 0 + 0.55c = 0.55c,\ v_{2i} = \frac{v_{2i}' + u}{1 + v_{2i}' u/c^2} = -0.340 c $$
	Using the same steps as above: \\
	$$ v_{1f} = 0.069c,\ v_{2f} = 0.703c $$
	Momentum: \\
	$$ p_i = m_1 v_{1i} + m_2 v_{2i} = 2m(0.550c) + m(-0.340c) = 0.76mc $$
	$$ p_f = m_1 v_{1f} + m_2 v_{2f} = 2m(0.069c) + m(0.703c) = 0.841mc $$

	\textbf{\emph{Momentum is not conserved?}}
\end{question}
Need a new definition of momentum that: \\
\begin{enumerate}
	\item Reduces to classical version p = mv at low speeds ($v << c$)
	\item Conserves momentum such that principle of relativity holds, if momentum is conserved at one reference frame it must be conserved at all reference frames.
\end{enumerate}
\newpage
\begin{result}[Relatavistic Interpretation]
	m: mass of particle \\
	v: velocity of particle as measured in a particular reference frame\\
	$$ p = \frac{mv}{\sqrt{1 - v^2/c^2}} (= \gamma mv) $$

	From above, for a stationary observer:\\
	$$ p_i' = \frac{m_1 v_{1i}'}{\sqrt{1-v_{1i}'^2/c^2}} + \frac{m_2 v_{2i}'}{\sqrt{1-v_{2i}'^2/c^2}} = -1.134 mc $$
	Note, the total initial momentum is different than in above classical interpretation. \\
	Now, after finding final velocity $v_{1f}' = 0.585c, v_{2f}' = 0.294c$ \\
	$$ p_f' = \frac{m_1 v_{1f}'}{\sqrt{1-v_{1f}'^2/c^2}} + \frac{m_2 v_{2f}'}{\sqrt{1-v_{2f}'^2/c^2}} = -1.134 mc $$

	Using the same steps as above to find the initial and final velocities in reference frame O: \\
	$$v_{1i} = 0.55c,\ v_{2i} = -0.340c$$
	$$ v_{1f} = 0.069c,\ v_{2f} = 0.703c $$
	Initial momentum $p_i = 0.956 mc$
	Final momentum $p_f = 0.956 mc$

	\textbf{\emph{Momentum is conserved in all reference frames}}
\end{result}

\begin{question}[Comparison of Classical and Relatavistic Momentum]
	Consider the momentum of an electron (m=9.11 x $10^{-31}kg$) moving with speed v = 0.75c. How does the relatavistic momentum compare with the classical result?
	\begin{answer}[Answer]
		$$ p = mv = 2.05 \cdot 10^{-22} kg m/s $$
		$$ p = \gamma mv = 3.1 \cdot 10^{-22}kg m/s $$
		Relativistic energy is about 50\% larger!
	\end{answer}
\end{question}

\subsubsection{Relativistic Kinetic Energy}
Similar as momentum, the classical definition of kinetic energy would lead to an apparetnt lack of energy conservation. 
$$ KE = \frac{1}{2}mv^2 $$
We need a relativistic definition for kinetic energy.\\

Consider the work-energy theorem (change in kinetic energy is equal to the integral of the force):
$$ W = \int F\ dx  = int (\frac{dp}{dt})\ dx = \int \frac{dx}{dt}\ dp = \int v\ dp $$
Using integration by parts:
$$ W = pv - \int p\ dv $$
Therefore,
$$ KE = \gamma mv \cdot v - \int_0^v \gamma mv\ dv $$
$$ = \gamma mv^2 + mc^2 \gamma - mc^2 $$
$$ KE = \frac{mc^2}{\sqrt{1-v^2/c^2}} - mc^2 $$

\subsubsection{Rest Energy}
We can write relativistic kinetic energy in terms of relativistic total energy and rest energy, $E$ and $E_0$.
$$ K = E - E_0,\  E = \gamma mc^2,\ E_0 = mc^2 $$
m: rest mass, mass as measured in rest frame of particle.

\begin{question}[Rest Energy]
	What is the rest energy of an electron (M = $9.11 \times 10^{-31} kg$)? What is the momentum of an electron moving with speed v = 0.75c?
	\begin{answer}[Answer]
		$$E_0 = mc^2 = (9.11 \times 10^{-31} kg) (3.00 \times 10^8 m/s)^2 $$
		$$ = 9.19 \cdot 10^{-14} kg m^2/s^2$$
		Unit: eV: 1 eV = 1.609 $\cdot 10^{-19}$
		$$E_0 = 5.11 \cdot 10^5 eV = 0.511 MeV$$
		Electron mass: $m_e = 0.511 MeV/c^2$
		\bigskip
		$$ p = \gamma mv = \frac{0.511 MeV/c^2 0.75 c}{1 - 0.75^2} = 0.58 MeV/c^2 c = 0.58 MeV/c $$
	\end{answer}
\end{question}

\begin{result}[Units:]
	These units are very convenient when working with particles at relativistic speeds. \\\\
		\begin{tabular}{l|l}
			Energy: & $MeV$ \\
			Momentum: & $MeV/c$ \\
			Mass: & $MeV/c^2$
		\end{tabular}
\end{result}

\newpage
\subsubsection{Energy-Momentum Relationship}
It is often useful to write the relativistic total energy of a particle in terms of its momentum:\\
\begin{align*}
	p &= mv \gamma \\
	E &= mc^2 \gamma \\
	E^2 &= \frac{(mc^2)^2)}{1-v^2/c^2}  \\
	(pc)^2 &= \frac{m^2v^2c^2}{1-v^2/c^2} \\
	E^2 &= \frac{(mc^2)^2 + m^2v^2c^2 - m^2v^2c^2}{1-v^2/c^2} \\
	&= \frac{(mc^2)^2(1-v^2/c^2)}{(1-v^2/c^2)} + (pc)^2
\end{align*}
\begin{result}[Energy-Momentum Relationship]
	$$ E^2 = (pc)^2 + (mc^2)^2 $$
\end{result}
\end{document}

