\documentclass{article}
\usepackage[margin=0.5in]{geometry}
\usepackage{titlesec}
\usepackage{graphicx}
\usepackage{enumitem}
\usepackage{ifthen}
\usepackage{fancyhdr}
\usepackage{xcolor}

% -------- %
% SECTIONS %
% -------- %
\newcounter{problemnumber}\setcounter{problemnumber}{1}
\titlespacing\section{0pt}{10pt}{0pt}   % Spacing between Problems
\titlespacing\subsection{0pt}{5pt}{0pt} % Spacing between Parts
\newcommand{\problem}[1][-1]{
    \setcounter{partnumber}{1}
    \ifnum#1>0
        \setcounter{problemnumber}{#1}
    \fi
    \section*{Problem \arabic{problemnumber}}
    \stepcounter{problemnumber}
}

\newcounter{partnumber}\setcounter{partnumber}{1}
\newcommand{\ppart}[1][-1]{
    \ifnum#1>0
        \setcounter{partnumber}{#1}
    \fi
    \subsection*{Part \parttype{partnumber}}
    \stepcounter{partnumber}
}

\newenvironment{question}{
    \color{gray}\itshape
    \vspace{5pt}
    \begin{tabular}{|p{0.97\linewidth}}
}{
    \end{tabular}\\[5pt]
}



% ------------- %
% HEADER/FOOTER %
% ------------- %
\setlength\parindent{0pt}
\setlength\headheight{30pt}
\headsep=0.25in
\pagestyle{fancy}
\lhead{\ifthenelse{\thepage=1}
    {\textbf{Trevor Smith} \\ \textbf{\writeday}}
}
\chead{\ifthenelse{\thepage=1}
    {\textbf{\huge{HOMEWORK \hwnumber}}}
    {\textbf{\large{HOMEWORK \hwnumber}}}
}
\rhead{\ifthenelse{\thepage=1}
    {\textbf{{\course}} \\ \textbf{Professor {\prof}}}
}
\cfoot{\thepage}
\renewcommand\headrulewidth{0.4pt}
\renewcommand\footrulewidth{0.4pt}



% ---------- %
% PARAMETERS %
% ---------- %
% \PARTTYPE:
% \Alph   := "Part A, Part B,  ..."
% \alph   := "Part a, Part b,  ..."
% \arabic := "Part 1, Part 2,  ..."
% \Roman  := "Part I, Part II, ..."
\newcommand\parttype{\Roman}

% \COURSE:
\newcommand\course{PHYS 2303}

% \HWNUMBER
\newcommand\hwnumber{9}

% \SEMESTER
\newcommand\semester{Spring 2021}

% \PROF
\newcommand\prof{Skinnari}

% \WRITEDAY
% \today is date of compilation, replace if writing due date rather than write date
\newcommand\writeday{\today}



%  ------- %
% DOCUMENT %
% -------- %
\begin{document}

\problem
\begin{question}
	(a) Determine the binding energy per ion pair in LiF from the cohesive energy. \\
	(b) Determine the binding energy per ion pair in LiF from the minimum potential energy at equilibrium separation (i.e. based on lattice parameters). \\
\end{question}
A) $B = \frac{E_{coh}}{N_A}$ \\
$B = 1030\ kJ/mol/6.02 \cdot 10^{23}\ ions/mol = 10.7 eV$\\

B)
$$B = \frac{\alpha e^2}{4\pi\epsilon_0 R_0} \left(1 - \frac{1}{n} \right) $$
$$B = \frac{1.7476 \cdot 1.44\ eV \cdot nm}{0.201 nm} \left(1 - \frac{1}{6} \right) $$
$$ B = 10.433 eV $$

\problem
\begin{question}
	Copper has a density of 8.96 g/cm3 and molar mass of 63.5 g/mol. Calculate the center-to-center distance between copper atoms in the fcc structure.
\end{question}

$ 8.96 g/cm^3 \cdot \frac{1 mol}{64.5 g} = 138914.729 mol / m^3 = 8.365 \cdot 10^{28}\ atoms/m^3$
The fcc structure contains 4 atoms, 1/2 per face and 1/8 per corner. If $l$ is the length of one edge of the cube, then $5 atoms/l^3 = 8.365 \cdot 10^{28} atoms/m^3$. Therefore,
$$ l = \left( \frac{5\ atoms}{8.365 \cdot 10^{28}\ atoms/m^3} \right)^{1/3} = 0.391 nm $$

\newpage
\problem
\begin{question}
Follow these steps to estimate the electrical conductivity of silver at temperature T = 300 K.
The density of silver is ρ = 10.5 g/cm³ and the molar mass is M = 108 g/mol. \\
(a) Determine the concentration of conduction electrons in silver assuming that each silver atom
contributes one electron. \\
(b) Use the equipartition of energy theorem (see e.g. Chapter 1 if you are unfamiliar with this) to
estimate the average speed of conduction electrons. \\
(c) Estimate the mean free path of conduction electrons by equating it to the nearest-neighbor
distance in the lattice. The latter must be estimated based on the numbers provided and the fact that
metallic silver forms an fcc lattice. \\
(d) Combine the results of (a-c) to calculate the electrical conductivity. 
\end{question}
A)\\ If we treat the free electrons as a gas, given that there is one free electron per Au atom (the two 5s electrons instead fill the 4d level, leaving one left over),
$$ 10.5\ g/cm^3 \cdot 1\ mol/108\ g = 0.0972\ mol\ electrons/cm^3 $$
B)\\ $$ v_{av} = \sqrt{\frac{3kT}{m}} = \sqrt{\frac{3(0.0258\ eV)}{0.511 \cdot 10^6 eV/c^2}} = 116,675.785\ m / s = 1.167 \cdot 10^5\ m/s $$
C) \\
$ 10.5 g/cm^3 \cdot \frac{1 mol}{108 g} = 97222 mol / m^3 = 5.855 \cdot 10^{28}\ atoms/m^3$
Given that this is fcc just like problem 2, 
$$ l = \left( \frac{5\ atoms}{5.855 \cdot 10^{28}\ atoms/m^3} \right)^{1/3} = 0.440 nm $$
D) \\
$$ \sigma = \frac{ne^2\tau}{m} $$
$ \tau = l/v_{av} = 0.440\ nm/1.167 \cdot 10^5\ m/s = 3.77 \cdot 10^{-15} s$ \\
$ n = 5.855 \cdot 10^{28}\ cm^{-3} $ \\
$ m = 9.109 \cdot 10^{-31}\ kg $ \\
$$ \sigma = \frac{(5.855 \cdot 10^{28}\ m^{-3})(1.60 \cdot 10^{-19}\ C)^2(3.77 \cdot 10^{-15} s)}{9.109 \cdot 10^{-31}\ kg} = 6.203\ MS/m $$


\newpage
\problem
\begin{question}
(a) Calculate the total nuclear binding energy of ³He and ³H. \\
(b) Account for any difference of the calculated nuclear binding energies by considering the
Coulomb interaction of the extra proton of ³He. 
\end{question}

A) The total nuclear binding energy is the difference in mass between the constituent particles and the whole atom.  \\
$^3He$:\\
$$ B = \left( 1(939.57\ MeV/c^2) + 2(938.28\ MeV/c^2) - 2809.4132 MeV / c^2 \right) c^2 =  6.716 MeV $$
$^3H$:\\
$$ B = \left( 2(939.57\ MeV/c^2) + 1(938.28\ MeV/c^2) - 2809.4132 MeV / c^2 \right) c^2 =  7.988 MeV $$

B) The difference is due to the two protons in He repelling each other.


\problem
\begin{question}
	Ordinary potassium contains 0.012\% of the naturally occurring radioactive isotope ⁴⁰K, which has a
half-life of 1.3×10⁹ year. \\
(a) What is the activity of 1.0 kg of potassium? \\
(b) What would have been the fraction of ⁴⁰K in natural potassium 4.5×10⁹ years ago?
\end{question}
A) \\
$$ t_{1/2} \approx \frac{0.693}{\lambda} \rightarrow \lambda = \frac{0.693}{1.3 \cdot 10^9 year} = 1.689 \cdot 10^{-17} Hz $$
1 kg of potassium is $1\ kg \cdot \frac{1\ mol}{39.098 g} = 25.75 mol\ K$. 25.75mol potassium will contain $25.75\ mol \cdot 0.00012\ mol\ ^{40}K/mol\ K = 0.00307\ mol\ ^{40}K$. 
$$a = \labmda N = 1.689 \cdot 10^{-17} Hz \cdot 0.00307\ mol \cdot A_N = 31,218\ molecules/second $$

B) \\
$$ N = N_0 e^{-\lambda t} = 0.00307\ mol \cdot e^{-1.689 \cdot 10^{-17}\ Hz \cdot (-4.5 \cdot 10^9\ years)} = 0.0338\ moles $$
The mole fraction is then 0.0338/25.75 = 0.13 \%.

\problem
\begin{question}
	What is the kinetic energy of the alpha particle that is emitted in the decay of the uranium isotope
with mass number A = 234?
\end{question}
$$ \ce{^{234}U} \rightarrow \ce{^{230}Th} + \ce{^4He} + KE $$
Replacing each element with its total energy (assuming the main atom is stationary before and after decay):
$$ 221,722.53 MeV/c^2 \rightarrow 216,141.7 MeV/c^2 + 3,728.4 MeV/c^2 + KE/c^2 $$
$$ c^2(221,722.53 MeV/c^2 - (216,141.7 MeV/c^2 + 3,728.4 MeV/c^2)) = 1,852 MeV $$


\newpage
\problem 
\begin{question}
Compare the recoil energy of a nucleus of mass 200 u that emits: \\
(a) a 5.0-MeV alpha particle, and \\
(b) a 5.0-MeV gamma ray?
\end{question}
A) \\
The speed of an alpha particle with 5 MeV KE is (wolfram alpha) 0.05174 c. The momentum will be 192.9 MeV/c. By conservation of momentum, the nucleus will also recoil with the same energy. This one will not be even remotely relativistic, but it will lose 2 Amu by ejecting the alpha particle. 192.9 MeV/c / 198 u = 0.001046 c. Using the equation for kinetic energy, we get 0.1009 MeV.

B) \\
The momentum of a 5 MeV photon will be 5 MeV/c. Using the same calculation as above, we get KE = 67.8 eV.

\newpage
\problem
\begin{question}
	A fossil bone fragment contains 50 g of carbon and registers a = 5 decays/s of carbon-14 activity.
Assume that the carbon-14 to carbon-12 isotope ratio in the atmosphere, and consequently within
living organisms, is constant over time at 1.3×10 ¹². Because carbon-14 decays radioactively with a
half-life of 1.8×10¹¹ s, the ratio of carbon-14 to carbon-12 in dead organisms decrease over time.\\
(a) How many carbon-14 isotopes were in the bone fragment when the animal was alive?\\
(b) Find the decay constant ($\lambda$) of carbon-14.\\
(c) Determine the age of the bone fragment. 
\end{question}
A)\\
The 50 g of carbon will be split by mole ratio between 14 and 12 is $1.3 \cdot 10^{-12}\ mol\ Carbon-14/1\ mol\ Carbon-12$. The average molecular weight will be (1-1.3e-12)*12 + (1.3e-12)*12 ok yeah it's 12, rounding to any number of meaningful digits.
$$ 50\ g\ carbon \cdot \frac{1\ mol\ C}{12\ g} \cdot \frac{1.3e-12\ mol\ ^{14}C}{1\ mol\ ^{12}C} = 3.262 \cdot 10^{12}\ molecules\ Carbon-13 $$

B)\\
$$t_{1/2} = 1/\lambda ln(2) $$
$t_{1/2} = 1.8 \cdot 10^{11}\ s$\\
$\lambda = 3.85081767e-12 hertz$

C)\\
We can find the N number of molecules currently by the expression relating activity, N, and $\lambda$.
$$ a = -\frac{dN}{dt} = \lambda N $$
$$ 3.85081767 \cdot 10^{-12} hertz \cdot N = 5\ decays/s \rightarrow N = 1.29842554 \cdot 10^{12}\ molecules\ ^{14}C $$
By
$$ N = N_0e^{-\lambda t} $$
given labmda, $N$, and $N_0$,
$$ 1.29842554 \cdot 10^{12} = 3.262 \cdot 10^{12}\cdot e^{-3.85081767 \cdot 10^{-12} t} $$
$$ \lambda t = ln \left(\frac{N}{N_0} \right) \rightarrow t = 7,580\ years $$



\end{document}
