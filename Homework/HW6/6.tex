\documentclass{article}
\usepackage[margin=0.5in]{geometry}
\usepackage{titlesec}
\usepackage{graphicx}
\usepackage{ifthen}
\usepackage{fancyhdr}
\usepackage{xcolor}

% -------- %
% SECTIONS %
% -------- %
\newcounter{problemnumber}\setcounter{problemnumber}{1}
\titlespacing\section{0pt}{10pt}{0pt}   % Spacing between Problems
\titlespacing\subsection{0pt}{5pt}{0pt} % Spacing between Parts
\newcommand{\problem}[1][-1]{
    \setcounter{partnumber}{1}
    \ifnum#1>0
        \setcounter{problemnumber}{#1}
    \fi
    \section*{Problem \arabic{problemnumber}}
    \stepcounter{problemnumber}
}

\newcounter{partnumber}\setcounter{partnumber}{1}
\newcommand{\ppart}[1][-1]{
    \ifnum#1>0
        \setcounter{partnumber}{#1}
    \fi
    \subsection*{Part \parttype{partnumber}}
    \stepcounter{partnumber}
}

\newenvironment{question}{
    \color{gray}\itshape
    \vspace{5pt}
    \begin{tabular}{|p{0.97\linewidth}}
}{
    \end{tabular}\\[5pt]
}



% ------------- %
% HEADER/FOOTER %
% ------------- %
\setlength\parindent{0pt}
\setlength\headheight{30pt}
\headsep=0.25in
\pagestyle{fancy}
\lhead{\ifthenelse{\thepage=1}
    {\textbf{Trevor Smith} \\ \textbf{\writeday}}
}
\chead{\ifthenelse{\thepage=1}
    {\textbf{\huge{HOMEWORK \hwnumber}}}
    {\textbf{\large{HOMEWORK \hwnumber}}}
}
\rhead{\ifthenelse{\thepage=1}
    {\textbf{{\course}} \\ \textbf{Professor {\prof}}}
}
\cfoot{\thepage}
\renewcommand\headrulewidth{0.4pt}
\renewcommand\footrulewidth{0.4pt}



% ---------- %
% PARAMETERS %
% ---------- %
% \PARTTYPE:
% \Alph   := "Part A, Part B,  ..."
% \alph   := "Part a, Part b,  ..."
% \arabic := "Part 1, Part 2,  ..."
% \Roman  := "Part I, Part II, ..."
\newcommand\parttype{\Roman}

% \COURSE:
\newcommand\course{PHYS 2303}

% \HWNUMBER
\newcommand\hwnumber{6}

% \SEMESTER
\newcommand\semester{Spring 2021}

% \PROF
\newcommand\prof{Skinnari}

% \WRITEDAY
% \today is date of compilation, replace if writing due date rather than write date
\newcommand\writeday{\today}



%  ------- %
% DOCUMENT %
% -------- %
\begin{document}
\problem
\begin{question}
	An electron is trapped inside an infinite potential energy well (one dimensional). The width of the potential energy well is L=0.300 nm.\\
(a) If the electron is in the ground state (n=1), what is the probability of finding it within 0.100 nm of the left-side wall? \\
(b) What is the probability if the electron is instead in the 99th excited state (n=100)? \\
(c) What probability would you expect for a classical particle? How does this compare with your results from (a) and (b)? This is an example of the so-called correspondence principle.
\end{question}
A)\\
The wave function for the particle is 
$$ \psi(x) = \frac{2}{L} sin \left(\frac{n \pi}{L}x \right) $$ 
Here:\\
L = 0.3 nm \\
n = 1 \\
The probability of finding the particle in a given region from $x_0\ to\ x_1$ is

$$ \int_{x_0}^{x_1} |\psi(x)|^2 = \frac{2}{L} \int_{x_0}^{x_1} sin^2 \left(\frac{n \pi}{L}x \right)\ dx $$ 
Where $x_0$ = 0 nm and $x_1$ = 0.1 nm. We can use the sine power-reducing formula:
$$ \int_{x_0}^{x_1} |\psi(x)|^2 = \frac{2}{2L} \int_{0}^{0.1} 1 - cos \left(\frac{2 n \pi}{L}x \right)\ dx $$ 
$$ = \frac{1}{L} \left[ \int_{0}^{0.1} 1\ dx  - \int_{0}^{0.1} cos \left(\frac{2 n \pi}{L}x \right)\ dx \right] $$ 
$$ = \frac{1}{L} \left[ x - \frac{L}{2n \pi} sin \left(\frac{2 n \pi}{L}x \right) \right] \rvert^{.1}_0 $$ 
$$ = \frac{1}{0.3 nm} \left[ x - \frac{0.3 nm}{2 \pi} sin \left(\frac{2 \pi}{0.3 nm}x \right) \right] \rvert^{.1}_0 $$ 
$$ = 1/3 - \frac{1}{2 \pi} sin \left(\frac{2 \pi}{3} \right) = 0.1955 $$ 

B)\\
We can just change n to 100 at the end.
$$ = \frac{1}{0.3} \left[ x - \frac{0.3}{2* 100 \pi} sin \left(\frac{2 * 100 \pi}{0.3}x \right) \right] \rvert^{.1}_0 $$ 
$$ = \frac{1}{3} - \frac{1}{2* 100 \pi} sin \left(\frac{2 * 100 \pi}{3} \right) \right] = 0.3320 $$ 

C)\\
In classical mechanics the probability of finding the particle in a length equal to 1/3 the length of the box is 1/3. The result from part A does not correspond with this probability, but the result from part B does.


\newpage
\problem
\begin{question}
A particle is confined to a three-dimensional infinite potential energy well. Sketch an
energy level diagram that shows the energies (in terms of E ), quantum numbers, and
degeneracies for the lowest 10 energy levels for this particle. 
\end{question}


\begin{figure}[h!]
	\centering
	\includegraphics[width=.8\linewidth]{./degenerates.png}
	\caption{Setup 3}
\end{figure}

\newpage
\problem
\begin{question}
Consider the quantum harmonic oscillator and its ground state wave function:\\
(a) Find the average value $<x>$ and the average value $<x^2>$.\\
(b) Find the uncertainty ∆x using the above. (Hint: see also textbook, end of Section 4.4
+ “extra” notes for Lecture 17 in Canvas.)
\end{question}
The wave function of the ground state is
$$ \psi(x) = A e^{-\alpha x^2} $$
Where
$$ A = \frac{m\omega_0}{\bar{h}\pi}^{1/4}  $$
$$ \alpha = \frac{\omega_0}{2\bar{h}} $$
THe average value of x is:
$$ \int_{-\infty}^{\infty} x \psi(x)^2 \ dx$$
$$ \int_{-\infty}^{\infty} x A e^{-2 \alpha x^2}\ dx $$
This is an odd function, and the integral will evaluate to zero. \\

The average value of $x^2$ is:
$$ \int_{-\infty}^{\infty} x^2 \psi(x)^2 \ dx$$
$$ \int_{-\infty}^{\infty} x^2 A^2 e^{-2 \alpha x^2}\ dx $$
Where $u = \sqrt{2 \alpha} x $, $du = \sqrt{2 \alpha} dx $ \\
$x = \frac{u}{\sqrt{2 \alpha}} $, $dx = \frac{du}{\sqrt{2 \alpha}} $ \\
$$ \frac{A^2}{(\sqrt{2 \alpha})^3} \int_{-\infty}^{\infty} u^2 e^{-u^2}\ dx $$
the value of the integral is equal to $\frac{\sqrt{\pi}}{4}$. This gives:
$$ \frac{A^2}{(2 \alpha)^{3/2}} \frac{\sqrt{\pi}}{4}$$
$$ \left(\frac{m\omega_0}{\bar{h}\pi}\right)^{1/2}  \left(\frac{\omega_0}{2\bar{h}} \right)^{-3/2}  \frac{\sqrt{\pi}}{4}$$ 
$$ \left(\frac{m\omega_0}{\bar{h}\pi}\right)^{1/2}  \left(\frac{2\bar{h}}{\omega_0} \right)^{3/2}  \frac{\sqrt{\pi}}{4}$$
$$ \frac{\bar{h}}{\omega_0} \left(\frac{m}{\pi}\right)^{1/2}  \left(2 \right)^{3/2}  \frac{\sqrt{\pi}}{4}$$
$$ \frac{\bar{h}}{\omega_0} \sqrt{\frac{m}{2}} $$

B)\\
$$ \Delta x = \sqrt{<x^2> - <x>^2} $$
$$ \Delta x = \sqrt{\frac{\bar{h}}{\omega_0} \sqrt{\frac{m}{2}} - 0} $$
$$ \Delta x = \sqrt{\frac{\bar{h}}{\omega_0} \sqrt{\frac{m}{2}}} $$



\newpage
\problem
\begin{question}
Consider again the quantum harmonic oscillator. The first excited state of the harmonic
oscillator has a wave function of the form:
$$ \psi(x) = A x e^{-ax^2} $$
(a) Similarly as we did in class for the ground-state wave function, find the constant a in the exponential, and the energy E for this (first excited state) wave function. \\
(b) Find the constant A for the wave function using the normalization condition.
\end{question}
$$ \frac{d\psi}{dx} = Ax(-2ax) e^{-ax^2} + A e^{-ax^2}  $$
$$ \frac{d\psi}{dx} = -2Aax^2 e^{-ax^2} + A e^{-ax^2}  $$

$$ \frac{d^2\psi}{dx^2} = -4 Aax e^{-ax^2} + 4Aa^2x^3 e^{-ax^2} - 2Aax e^{-ax^2} $$
$$ \frac{d^2\psi}{dx^2} = -6 Aax e^{-ax^2} + 4Aa^2x^3 e^{-ax^2} $$
$$ \frac{d^2\psi}{dx^2} = A(-6 a + 4a^2x^2)x e^{-ax^2} $$
The Schrödinger equation is:
$$ - \frac{\bar{h}^2}{2m} \frac{d^2\psi}{dx^2} + 1/2\ kx^2 \psi(x) = E\psi(x) $$
Substitute into Schrödinger equation:
$$ - \frac{\bar{h}^2}{2m}A(-6 a + 4a^2x^2)x e^{-ax^2} + 1/2\ kx^2 A x e^{-ax^2} = EA x e^{-ax^2} $$
The exponentials cancel out on both sides of the equation.
$$ - \frac{\bar{h}^2}{2m}(-6 a + 4a^2x^2) + 1/2\ kx^2  = E $$
We want a solution that is valid for ANY value of x:
$$  \frac{\bar{h}^2 6a}{2m} - \frac{\bar{h}^2}{2m} 4a^2x^2 + 1/2\ kx^2  = E $$
$$ - \frac{\bar{h}^2}{2m} 4a^2x^2 + 1/2\ kx^2  = E - \frac{\bar{h}^2 6a}{2m} $$
$$ \left(1/2\ k- \frac{\bar{h}^2}{2m} 4a^2x^2 \right)x^2  = E - \frac{\bar{h}^2 6a}{2m} $$
$$ \left(1/2\ k- \frac{\bar{h}^2}{2m} 4a^2x^2 \right)x^2  = E - \frac{\bar{h}^2 6a}{2m} $$

Each side must be equal to zero
$$ \frac{k}{2} = \frac{\bar{h}^2}{2m} 4a^2 $$
$$ \frac{km}{4\bar{h}^2} = a^2 $$
$$ a = \frac{\sqrt{km}}{2\bar{h}} $$
$$ E = \frac{\bar{h}^2 6a}{2m} $$
$$ E = \frac{\bar{h}^2 6}{2m}\frac{\sqrt{km}}{2\bar{h}}  $$
$$ E =3/2 \bar{h} \sqrt{\frac{k}{m}}  $$
\\
\newpage
B)\\
\[ 1 = \int_{-\infty}^\infty |\psi(x)|^2\ dx \]
\[ 1 = A^2 \int_{-\infty}^\infty x^2 e^{-2ax^2}\ dx \ \ \ where\ a = \frac{\sqrt{km}}{2\bar{h}} \]
\[ 1 = A^2 \int_{-\infty}^\infty x^2 e^{-2ax^2}\ dx \ \ \ where\ a = \frac{\sqrt{km}}{2\bar{h}} \]

Where $u = \sqrt{2 \alpha} x $, $du = \sqrt{2 \alpha} dx $ \\
And $x = \frac{u}{\sqrt{2 \alpha}} $, $dx = \frac{du}{\sqrt{2 \alpha}} $ \\
$$ 1 =  \frac{A^2}{(\sqrt{2 \alpha})^3} \int_{-\infty}^{\infty} u^2 e^{-u^2}\ dx $$
$$ 1 =  \frac{A^2}{(\sqrt{2 \alpha})^3} \frac{\sqrt{\pi}}{4} $$
$$ A^2 =  \frac{4(\sqrt{2 \alpha})^3}{\sqrt{\pi}} $$
$$ A =  \sqrt[4]{\frac{2}{\pi}} (\sqrt[4]{\alpha})^3 $$
$$ a = \frac{\omega_0}{2\bar{h}} $$
$$ A =  \sqrt[4]{\frac{2}{\pi}} \left(\sqrt[4]{\frac{\omega_0}{2\bar{h}}}\right)^3 $$
$$ A = \sqrt{\pi/2} \left({\frac{2}{\pi}}\right)^{3/4} \left(\frac{\omega_0}{2\bar{h}}\right)^{3/4} $$
$$ A = \sqrt{\pi/2} \left(\frac{\omega_0}{\bar{h}\pi}\right)^{3/4} $$


\newpage
\problem
\begin{question}
	Find the normalization constants B, D in terms of A by applying the 
	boundary conditions at x=0 for the given equations in the situation of 
	$E < U_0$.
\end{question}
$$\psi_0(0) = B,\ \ \ \psi_1(0) = D $$
B = D\\
$$ \frac{d\psi_0}{dx} = A k_0 cos(k_0x) - B k_0 sin(k_0x) $$
$$ \frac{d\psi_1}{dx} = -k_1 D e^{-k_1x} $$
$$ \psi'_0(0) = A k_0 $$
$$ \psi'_1(0) = - Dk_1 $$
These derivatives must be equivalent
$$ D = B = - A \frac{k_0}{k_1} $$

\problem
\begin{question}
A beam of electrons, each with kinetic energy K = 9.0 keV, is incident on a potential
energy step of U = 5.0 keV. What fraction of the beam is reflected by the potential?
\end{question}
$$ x < 0:\ \Psi_0(x,t) = A'e^{i(k_0x-\omega t)} + B'e^{-i(k_0x+\omega t)} $$
$$ x \geq 0:\ \Psi_1(x,t) = C'e^{i(k_1x-\omega t)} $$
D is 0 because it tends towards infinity. \\

At x = 0 the equations must be equivalent, therefore: \\
$$ A + B = C $$
The derivatives must also be equivalent:
$$ k_0 (A - B) = k_1 C $$
Plugging in:
$$ k_0 A - k_0 B = k_1 A + k_1 B $$
$$ k_0 A - k_1 A = k_0 B + k_1 B $$
$$ A (k_0 - k_1) = B (k_0 + k_1) $$

The reflection coefficient will relate $k_0\ and\ k_1$ by $\frac{|A|^2}{|B|^2}$, where $k_1/k_0 = \frac{U_0 - E)}{\sqrt{E}}$:
$$ R = \left(\frac{\sqrt{E} - \sqrt{E - U_o}}{\sqrt{E} + \sqrt{E - U_o}} \right)^2 $$
Where E = 9.0 keV and U = 5.0 keV
$$ R = \left(\frac{\sqrt{9.0} - \sqrt{9.0 - 5.0}}{\sqrt{9.0} + \sqrt{9.0 - 5.0}} \right)^2 $$
$$ R = \frac{3 - 2}{3 + 2}^2 $$
$$ R = \frac{1}{5} ^2 = 1/25 $$

\end{document}
