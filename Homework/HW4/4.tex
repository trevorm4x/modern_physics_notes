\documentclass{article}
\usepackage[margin=0.5in]{geometry}
\usepackage{titlesec}
\usepackage{ifthen}
\usepackage{fancyhdr}
\usepackage{xcolor}

% -------- %
% SECTIONS %
% -------- %
\newcounter{problemnumber}\setcounter{problemnumber}{1}
\titlespacing\section{0pt}{10pt}{0pt}   % Spacing between Problems
\titlespacing\subsection{0pt}{5pt}{0pt} % Spacing between Parts
\newcommand{\problem}[1][-1]{
    \setcounter{partnumber}{1}
    \ifnum#1>0
        \setcounter{problemnumber}{#1}
    \fi
    \section*{Problem \arabic{problemnumber}}
    \stepcounter{problemnumber}
}

\newcounter{partnumber}\setcounter{partnumber}{1}
\newcommand{\ppart}[1][-1]{
    \ifnum#1>0
        \setcounter{partnumber}{#1}
    \fi
    \subsection*{Part \parttype{partnumber}}
    \stepcounter{partnumber}
}

\newenvironment{question}{
    \color{gray}\itshape
    \vspace{5pt}
    \begin{tabular}{|p{0.97\linewidth}}
}{
    \end{tabular}\\[5pt]
}



% ------------- %
% HEADER/FOOTER %
% ------------- %
\setlength\parindent{0pt}
\setlength\headheight{30pt}
\headsep=0.25in
\pagestyle{fancy}
\lhead{\ifthenelse{\thepage=1}
    {\textbf{Trevor Smith} \\ \textbf{\writeday}}
}
\chead{\ifthenelse{\thepage=1}
    {\textbf{\huge{HOMEWORK \hwnumber}}}
    {\textbf{\large{HOMEWORK \hwnumber}}}
}
\rhead{\ifthenelse{\thepage=1}
    {\textbf{{\course}} \\ \textbf{Professor {\prof}}}
}
\cfoot{\thepage}
\renewcommand\headrulewidth{0.4pt}
\renewcommand\footrulewidth{0.4pt}



% ---------- %
% PARAMETERS %
% ---------- %
% \PARTTYPE:
% \Alph   := "Part A, Part B,  ..."
% \alph   := "Part a, Part b,  ..."
% \arabic := "Part 1, Part 2,  ..."
% \Roman  := "Part I, Part II, ..."
\newcommand\parttype{\Roman}

% \COURSE:
\newcommand\course{PHYS 2303}

% \HWNUMBER
\newcommand\hwnumber{1}

% \SEMESTER
\newcommand\semester{Spring 2021}

% \PROF
\newcommand\prof{Skinnari}

% \WRITEDAY
% \today is date of compilation, replace if writing due date rather than write date
\newcommand\writeday{\today}



%  ------- %
% DOCUMENT %
% -------- %
\begin{document}
\problem
\begin{question}
Protons are incident on a copper foil that is 12 μm thick. Copper has a density of 8.95 g/cm³ and
a molar mass of 63.5 g/mole. \\
(a) What should the proton kinetic energy be in order for the distance of closest approach to
equal the nuclear radius (5.0 fm)? \\
(b) If the proton energy is 7.5 MeV, what is the impact parameter for scattering at 120 degrees? \\
(c) What is the minimum distance between the proton and nucleus for this case? \\
(d) What fraction of the protons is scattered beyond 120 degrees? \\
\end{question}

A) \\
$$ d = \frac{1}{4\pi\epsilon_0}\frac{zZe^2}{K} $$
z = 1\\
Z = 29\\
d = $ 5 fm $\\
constant $= k_r\ (arbitrary\ name) =  \frac{1}{4\pi\epsilon_0} $ = 1.44 MeV * fm \\
$$ K = k_r \frac{zZ}{d} = 8.352 MeV $$ \\

B) \\
$$ b = \frac{zZ}{2K} \frac{e^2}{4\pi\epsilon_0} cot(\theta/2) $$
$\theta = $120 degrees\\
$$ b = 1.44 MeV \cdot fm \cdot \frac{29}{2 \cdot 7.5} cot(60) = 1.607 fm $$

C) \\
From equation 6.18,
$$ 1/2 mv^2 = 1/2 m\frac{v^2b^2}{r_{min}^2} + k_r \frac{zZ}{r_{min}} $$
This can be rewritten to solve for $r_{min}$:
$$ 0 = -1/2 mv^2 r_{min}^2 + k_r zZ r_{min} + 1/2mv^2b^2 $$
Where \\
$mv^2 = 7.5 MeV $ \\
b = 1.607 fm \\
z, Z = 1, 29 \\
$k_r = 1.44 MeV\ fm$ \\
We can work out the coefficients of the above quadratic equation. \\
$a = -1/2 mv^2 = -3.75 $
$b = k_r zZ = 1.44 *29 = 41.76 $
$c = 1/2 mv^2b^2 = 1/2 * 7.5 * 1.607 = 6.026 $
$$ r_{min} = \frac{-b +- \sqrt{b^2 - 4ac}}{2a} = \frac{-41.76 +- \sqrt{41.76^2 - 4 * -3.75 * 6.026}}{-7.5} = 11.28 fm $$

D) \\
Fraction that scatters above 120 degrees
$$ f(\theta > 120) = nt\pi b^2 $$
Where \\
$ n = \frac{\rho N_A}{M} = 8.96 g/cm^3 * 6.022×10^{23} / 63.55 g/mol = 8.49 \cdot 10^{22} /cm^3 $\\
t = $12 \mu m = 1.2 \cdot 10^{-3} cm$ \\
$b = 1.607 fm = 1.607 \cdot 10^{-13} cm $

$ f = 8.49 \cdot 10^{22} /cm^3 \cdot 1.2 \cdot 10^{-3} cm \cdot \pi \cdot (1.607 \cdot 10^{-13} cm)^2 = 8.26 \cdot 10^-6 $ 

\newpage
\problem
\begin{question}
Alpha particles with kinetic energy of 6.0 MeV are incident at a rate of 3.0×10⁷ per second on a
gold foil of thickness 3.0 µm. A circular detector of diameter 1.0 cm is placed 12.0 cm from the
foil at an angle of 30 degrees with the direction of the incident alpha particles. At what rate does
the detector measure scattered alpha particles?
\end{question}

$$ N(\theta) = \frac{nt}{4r^2} \cdot \left(\frac{zZ}{2d}\right)^2\cdot \left(\frac{e^2}{4\pi\epsilon_0} \right)^2\cdot\frac{1}{sin^4(\theta/2)} $$
where\\
n = number of nuclei per unit volume = density per unit volume * atoms per gram = $ 19.3 g/cm^3 \cdot \frac{1 au}{1.6605 \cdot 10^{-24}g} \cdot \frac{1 atom}{196 au} = 5.93 \cdot 10^{22} /cm^3 $\\
t = 3.0 $\mu m$ = .0003 cm \\
r = 12 cm \\
z, Z = 2, 79 \\
K = 6.0 MeV \\
$\theta = 30 $ degrees \\
$$ N(\theta) = \frac{.0003cm \cdot 5.93 \cdot 10^{22} atoms/cm^3}{4 * (12cm)^2} \cdot \left(\frac{2 \cdot 79}{2*6.0 MeV}\right)^2\cdot \left( 1.44 MeV \cdot 10^{-13} cm \right)^2\cdot\frac{1}{sin^4(30 degrees/2)} = 2.474 \cdot 10^-5 /cm^2 $$
This gives the probability for a particle to hit the detector per unit area. The area of the detector itself is $ \pi * .5cm^2 = 0.7854 cm^2 $, so the total probability for a given particle to hit the detector is $2.474 \cdot 10^{-5} \cdot 0.7854 = 1.94 \cdot 10^{-5}$. The rate of incident particles is $3.0 \cdot 10^7 particles/s$, therefore the rate of particles hitting the detector will be $3.0 \cdot 10^7 \cdot 1.94 \cdot 10^{-5} = 583$ particles per second.



\newpage
\problem
\begin{question}
One of the lines in the Brackett series (series limit = 1458 nm) has a wavelength of 1817 nm.
What is the next higher and next lower wavelengths in the series? 
\end{question}
$$ \lambda = \lambda_{limit} \frac{n^2}{n^2-n_0^2}  $$
$n_0 = 4$ for brackett series
$$ 1817 nm = 1458 nm \frac{n^2}{n^2-16} $$
$$ 0.802 = 1 - 16/n^2 $$
$$ -0.198 = -16/n^2 $$
$$ 5.06 = /n^2/16 $$
n = 9 \\
 \\
Next higher n = 10 //

$$ \lambda = 1458 \frac{10^2}{10^2-4^2}  $$
$$ \lambda = 1458 \frac{10^2}{84} = 545 nm  $$

Next lower n = 8 //

$$ \lambda = 1458 \frac{8^2}{8^2-4^2}  $$
$$ \lambda = 1458 \frac{8^2}{48} = 954 nm  $$

\newpage
\problem
\begin{question}
What is the ionization energy of: \\
(a) the n = 3 level of hydrogen? \\
(b) the n = 2 level of He+ (singly ionized helium)? \\
(c) the n = 4 level of Li++ (doubly ionized lithium)? \\
\end{question}
A)\\
The ionization energy is the same as the magnitude of the energy of the electron. 
$$ |E_3| = \left| \frac{-13.6 eV}{9} \right| = 1.51 eV $$

B)\\
$$ |E_n| = \left| \frac{-13.6 eV Z^2}{n^2} \right| = \left| \frac{-13.6 eV 2^2}{2^2} \right| = 13.6 eV $$

C)\\
$$ |E_n| = \left| \frac{-13.6 eV Z^2}{n^2} \right| = \left| \frac{-13.6 eV 3^2}{4^2} \right| = 7.65 eV $$



\newpage
\problem
\begin{question}
In the n=4 state of hydrogen, find the electron's velocity, kinetic energy, and potential energy. 
\end{question}
$ r_n = a_0 n^2 = 0.0529 nm * 4^2 = 0.8464 nm $ \\
The velocity of an electron is given by 
$$ rmv = n \bar{h} \rightarrow v = \frac{n \bar{h}}{rm} $$
Where \\
r = 0.8464 nm\\
n = 4\\
$m = 0.511 MeV/c^2$
$\bar{h} = 6.582 \cdot 10^-22 MeV s$
$$ v = \frac{4 \cdot 6.582 \cdot 10^{-22} MeV\ s}{0.8464 nm \cdot 0.511 MeV/c^2} = .00182 c $$
\\

Kinetic energy is given by $K = \frac{1}{2m} \left(\frac{n\bar{h}}{r}\right)^2$
$$K = \frac{1}{2 \cdot 0.511 MeV/c^2} \left(\frac{4 \cdot 6.582 \cdot 10^{-22} MeV\ s}{0.8464 nm} \right)^2 = 8.509 \cdot 10^{-7} MeV $$
\\

Potential energy is given by $U = \frac{-1}{4\pi\epsilon_0} \frac{e^2}{r} $\\
Where\\
e = $1.6 \cdot 10^{-19} C$
U = -1.6967 eV


\newpage
\problem
\begin{question}
Adjacent wavelengths 72.90 nm and 54.00 nm are found in one series of transitions among the
radiations emitted by doubly-ionized lithium. Find the value of n0 for this series and find the next
wavelength in the series.
\end{question}
For transition from $n\ to\ n_0$ in lithium, where Z = 3:\\
$$\lambda = \frac{hc}{\Delta E} = \frac{1240 eV \cdot nm}{(-13.6 eV)Z^2(1/n^2 - 1/n_0^2)} = 10.13 nm \cdot \frac{n^2n_0^2}{n^2-n_0^2} $$
We can just plug in some numbers for $n_0$, and then drag n up and down to see what works. For $n_0$ = 1 the range of wavelengths for any n is very small. However, for n = 2 72.90 nm pops right out. \\

$n_0$ = 2 and n = 3 for 72.9 nm, n = 4 for 54.0 nm. The next member of the series would be n = 5, giving 48.24 nm.


\problem
\begin{question}
A beam of alpha particles with kinetic energy of 6.0 MeV impinges on a 1.0 $\mu$m thick silver foil.
The beam current is 1.0 nA. How many alpha particles per second will be deflected by more than
45 degrees? (For silver: Z = 47, density = 10500 kg/m³, and molar mass = 108 g/mole.) 
\end{question}
$$ b = \frac{zZ}{2K} \frac{e^2}{4\pi\epsilon_0} cot(\theta/2) $$
Where \\
$k_r = \frac{e^2}{4\pi\epsilon_0} = 1.44 MeV\ fm$ \\
z = 2 \\
Z = 47 \\
$\theta = 45\ degrees$ \\
K = 6.0 MeV \\
b = $\frac{2*47}{2*6.0 MeV} \cdot k_r \cdot cot(22.5) = 27.23 fm$

$$ f = nt\pi b^2 $$
Where \\
t = $1.0 \mu m = 1.0 \cdot 10^9 fm $\\
$ n = \frac{\rho N_A}{M} = 1049 g/cm^3 * 6.022×10^{23} / 107.9 g/mol =  5.855 \cdot 10^{22} /cm^3 $\\
$$ f = 1.364 \cdot 10^{-4} $$
\\

The beam current of 1.0 nA corresponds to 1 nC/s, or $1 \cdot 10^{-9} \cdot 6.241 \cdot 10^{18} = 6.241 \cdot 10^9$ charges per second. However, alpha particles have a charge of +2, so the number of particles per second will be half that, or $3.121 \cdot 10^9$particles/second.\\

The number of particles per second deflected more than 45 degrees will be the fraction deflected times the total, or $3.121 \cdot 10^9 \cdot 1.364 \cdot 10^{-4} = 425,000 $ particles/second.


\end{document}
