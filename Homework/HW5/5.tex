\documentclass{article}
\usepackage[margin=0.5in]{geometry}
\usepackage{titlesec}
\usepackage{graphicx}
\usepackage{ifthen}
\usepackage{fancyhdr}
\usepackage{xcolor}

% -------- %
% SECTIONS %
% -------- %
\newcounter{problemnumber}\setcounter{problemnumber}{1}
\titlespacing\section{0pt}{10pt}{0pt}   % Spacing between Problems
\titlespacing\subsection{0pt}{5pt}{0pt} % Spacing between Parts
\newcommand{\problem}[1][-1]{
    \setcounter{partnumber}{1}
    \ifnum#1>0
        \setcounter{problemnumber}{#1}
    \fi
    \section*{Problem \arabic{problemnumber}}
    \stepcounter{problemnumber}
}

\newcounter{partnumber}\setcounter{partnumber}{1}
\newcommand{\ppart}[1][-1]{
    \ifnum#1>0
        \setcounter{partnumber}{#1}
    \fi
    \subsection*{Part \parttype{partnumber}}
    \stepcounter{partnumber}
}

\newenvironment{question}{
    \color{gray}\itshape
    \vspace{5pt}
    \begin{tabular}{|p{0.97\linewidth}}
}{
    \end{tabular}\\[5pt]
}



% ------------- %
% HEADER/FOOTER %
% ------------- %
\setlength\parindent{0pt}
\setlength\headheight{30pt}
\headsep=0.25in
\pagestyle{fancy}
\lhead{\ifthenelse{\thepage=1}
    {\textbf{Trevor Smith} \\ \textbf{\writeday}}
}
\chead{\ifthenelse{\thepage=1}
    {\textbf{\huge{HOMEWORK \hwnumber}}}
    {\textbf{\large{HOMEWORK \hwnumber}}}
}
\rhead{\ifthenelse{\thepage=1}
    {\textbf{{\course}} \\ \textbf{Professor {\prof}}}
}
\cfoot{\thepage}
\renewcommand\headrulewidth{0.4pt}
\renewcommand\footrulewidth{0.4pt}



% ---------- %
% PARAMETERS %
% ---------- %
% \PARTTYPE:
% \Alph   := "Part A, Part B,  ..."
% \alph   := "Part a, Part b,  ..."
% \arabic := "Part 1, Part 2,  ..."
% \Roman  := "Part I, Part II, ..."
\newcommand\parttype{\Roman}

% \COURSE:
\newcommand\course{PHYS 2303}

% \HWNUMBER
\newcommand\hwnumber{5}

% \SEMESTER
\newcommand\semester{Spring 2021}

% \PROF
\newcommand\prof{Skinnari}

% \WRITEDAY
% \today is date of compilation, replace if writing due date rather than write date
\newcommand\writeday{\today}



%  ------- %
% DOCUMENT %
% -------- %
\begin{document}
\problem
\begin{question}
	What is the de Broglie wavelength of:\\
	(a) A proton with an energy of 5 MeV? \\
	(b) A proton with an energy of 7 TeV? \\
\end{question}

The de Broglie Wavelength is calculated $\lambda = \frac{h}{p}$. \\
(a)\\
The rest energy of a proton is 938 MeV, so since rest energy > kinetic energy we can use the classical calculation for momentum, $p = mv$. \\
$$ KE = 5 MeV = 1/2 m v^2 = 1/2 \cdot 938 MeV/c^2 \cdot v^2 $$
$$ v^2 = 2 \frac{5 MeV}{938 MeV/c^2} $$
$$ v = 0.103 c $$
$$ p = mv = 938 MeV/c^2 * 0.103 c = 96.6 MeV/c $$
$$ \lambda = \frac{h}{p} = \frac{4.1357 MeV s}{96.6 MeV/c} = 1.28 * 10^{-14} m $$.

(b)\\
At these energies we are in the extreme relativistic domain. We can assume that the rest energy doesn't matter. \\

$$ KE = \frac{mc^2}{\sqrt{1-v^2/c^2}} $$
$$ KE = \gamma mc^2 $$
$$ 7.0 \cdot 10^6 MeV = \gamma 938 MeV $$
$$ \gamma = 7462.7 $$
$$ p = \gamma mc = 7462.7 \cdot 938 MeV/c^2 \cdot c = 7\ TeV/c $$
Okay, I was going to start with that assumption but wasn't sure. I guess I proved it.\\

$$ \lambda = \frac{hc}{pc} = \frac{1240 eV \cdot nm}{7.0 \cdot 10^{12} eV}
= 1.77 \cdot 10^{-19} m $$

\newpage
\problem
\begin{question}
	We wish to study atoms of diameter 2 Å (1 Å = 0.1 nm) using an electron microscope.\\
	(a) What should the electron de Broglie wavelength be?\\
	(b) What potential difference must the electrons be accelerated through to have this wavelength? \\
\end{question}
a)\\
The wavelength of the electrons should be the same as the diameter of the object, or 0.2 nm. \\

b)\\
$$ qV = KE{electron} $$
If we assume these speeds are non-relatvisitic, we can express KE as a function of momentum:
$$ qV = p^2/2m $$
$$ V = \frac{p^2}{2mq} $$
And we can then express p as $p = h/\lambda$.
$$ V = \frac{h^2}{\lambda^2\cdot 2 \cdot m \cdot q} $$
$$ V = \frac{(6.6\cdot10^{-34})^2}{(0.20 \cdot 10^{-9})^2 \cdot 2 \cdot 9.1 \cdot 10^{-31} \cdot 1.6 \cdot 10^{-19}} = 37.4 V $$

\problem
\begin{question}
	The $J/\psi$ meson, which was discovered in 1974, was measured to have an average mass
of 3.100 GeV/c² and an intrinsic width of 63 keV/c² (= spread in the measured
distribution of its rest energy). Estimate the lifetime of this particle using the
uncertainty principle.
\end{question}
The uncertainty principle is given as:
$$ \Delta E \Delta t = \frac{\bar{h}}{2} $$
$\Delta E = 63 keV = 6.3 \cdot 10^4 eV $ \\
$ \bar{h} = 6.58 \cdot 10^{-16} eV \cdot s $ \\
$$ \Delta t = \frac{6.58 \cdot 10^{-16} eV \cdot s}{2 \cdot 6.3 \cdot 10^4 eV} = 5.22 \cdot 10^{-21} seconds $$

\newpage
\problem
\begin{question}
(a) Find the de Broglie wavelength of a nitrogen molecule in air at room temperature
(293 K)? \\

(b) The density of air at room temperature and atmospheric pressure is 1.292 kg/m³.
Estimate the average distance between air molecules at this temperature and compare
with the de Broglie wavelength from (a). What do you conclude about the importance
of quantum effects in air at room temperature?
Note: Air predominantly consists of nitrogen, so in this problem we approximate air as
made up of nitrogen molecules. \\

(c) Estimate the temperature at which quantum effects might become important.
\end{question}
A)\\
The kinetic energy of a gas particle as a function of temperature is given by
$$ 3/2 k\cdot T $$ Where k is the Boltzmann constant $k = 8.76173 \cdot 10^{-5} eV/K $

$$ KE = 3/2 \cdot k \cdot 293 K = 0.0385 eV $$
We can ``convert" this to momentum via $p = \sqrt{KE * 2m}$ \\
Where m = 28 g/mol for 1/An moles = 28g/An = $26,081 MeV/c^2$
$$ p = \sqrt{0.0385 * 2 * 26,081 MeV/c^2} = 0.04481 MeV/c $$
The de Broglie wavelength is given by
$$ \labmda = \frac{hc}{pc} = \frac{1240 eV \cdot nm}{0.04481 MeV} = 0.02767 nm $$

B)\\
The experimental value of the radius of a Nitrogen atom is 65 pm, or a diameter of 130 pm. The mass of a nitrogen atom is 28 g/mol, which for 1 cubic meter of air weighing 1.292 kg gives us $1,292\ g \cdot \frac{1\ mol}{28\ g} = 46.14 mol = 2.7786 \cdot 10^{25}$ molecules. 

If the same number of molecules were tightly packed in a cubic arrangement they would take up $\sqrt[3]{2.7786 \cdot 10^{25} molecules} \cdot \frac{130 pm}{1 molecule} = 3.937 \cdot 10^{10} \ $picometers in each direction, or $ 6.105 \cdot 10^{-5} m^3$. This gives 1 cubic meter - that previous number, which is essentially 1, free space. The average free space is divied up equally to each atom.
This gives 
$$ \frac{1\ m^3}{2.7768 \cdot 10^{25}\ molecules} = 3.5989 \cdot 10^{-26}\ m^3/molecule $$
By taking the cube root of this number, we can estimate the average distance between molecules, 3.301 nm. This is two orders of magnitude greater than the de Broglie wavelength given above.

C)\\
We could guess that these quantum effects may become important when the de Broglie wavelenth is equal to the distance between molecules, where the de Broglie wavelength is given by
$$ \lambda = \frac{h}{\sqrt{(3/2)\cdot k \cdot T \cdot 2 \cdot m}} $$
$$ 3.3 nm = \frac{h}{\sqrt{(3/2)\cdot k \cdot T \cdot 2 \cdot m}} $$
$$ 3.3 nm = \frac{1240 eV\cdot nm/c}{\sqrt{(3/2)\cdot 8.76173 \cdot 10^{-5} eV/K \cdot T \cdot 2 \cdot 26081 MeV/c^2}} $$
$$ T \cdot (3.3 nm)^2 \cdot (3/2)\cdot 8.76173 \cdot 10^{-5} \cdot 2 \cdot 26081 MeV/c^2 = (1240 eV\cdot nm/c)^2 $$
$$ T = \frac{(1240 eV\cdot nm/c)^2}{(3.3 nm)^2 \cdot (3/2)\cdot 8.76173 \cdot 10^{-5} eV/K \cdot 2 \cdot 26081 MeV/c^2} $$
$$ T = 0.02 K $$



\newpage
\problem
\begin{question}
Two plane waves, y1 = 0.6cos(6x - 200t) and y2 = 0.6cos(5.8x - 180t), travel
simultaneously along a long wire. Variables $x$ and $y_1,y_2$ are measured in meters and t
in seconds.\\
(a) Write the wave function for the resultant wave $(y = y_1+y_2)$ in the form of Eq. 4.27
from the textbook (compare also lecture 13).\\
(b) Calculate the phase velocity of the resultant wave.\\
(c) Calculate the group velocity.\\
(d) Determine the range $\Delta x$ between successive zeros of the group and relate it to Δk. 
\end{question}
A)\\
The combined traveling wave can be represented as 
$$ y(x, t) = 0.6 cos(6x - 200 t) + 0.6 cos(5.8x - 180 t) $$
Using the formula
$$ y(x, t) = 2A cos \left( \frac{\Delta k}{2} x - \frac{\Delta \omega}{2}t \right) cos \left( \frac{k_1+k_2}{2} x - \frac{\omega_1+\omega_2}{2}t \right) $$
We can plug in numbers to get
$$ y(x, t) = 1.2 cos \left( \frac{0.2}{2} x - \frac{20}{2}t \right) cos \left( \frac{11.8}{2} x - \frac{380}{2}t \right) $$

B)\\
The phase velocity is given by $v_{phase} = \frac{\omega}{k}$ for any one wave in the wave packet. $v_{phase} = \frac{200}{6} = 33.3\ m/s$.

C)\\
The group velocity is given by $v_{group} = \frac{\Delta \omega}{\Delta k} = \frac{20}{.2} = 100 m/s $.

D)\\
The group behavior is governed by the first cosine in the expression, $cos(.1x - 10t)$. The phase diffence between consecutive zeroes is $\pi$, which implies that $(0.1x_2 - 10t_0) - (0.1x_1 - 10t_0) = \pi$, $0.1 \Delta x = \pi$, $\Delta x = \frac{\pi/0.1}$. We can substitude in $\Delta k = 0.2$ to get $2\pi = \Delta x \Delta k$.

\newpage
\problem
\begin{question}
A wave has the following form when x $<$ 0:
$$ y = A cos \left( \frac{2\pi x}{\lambda} + \frac{\pi}{3} \right) $$ 
For x $>$ 0, the wavelength is $\lambda/2$.\\
By applying continuity conditions at x = 0, find the amplitude (in terms of A) and the phase of the wave in the region x $>$ 0. Sketch the wave, showing both x $<$ 0 and x $>$ 0 regions. 
\end{question}
Boundary conditions state that the wave function and the slope must be continous. The derivative at x = 0 will be:
$$ \frac{\delta y}{\delta x} = - \frac{2\pi}{\lambda} A\ sin\left(\frac{2\pi x}{\lambda} + \frac{\pi}{3} \right) = -A \frac{\sqrt{3}\pi}{\lambda}\ at\ x = 0 $$
Therefore, at x = 0:
$$ - A_1 \frac{2\pi}{\lambda} \ sin\left(\frac{\pi}{3} \right) = - A_2 \frac{4\pi}{\lambda} sin(\phi) $$
As $\lambda_2 = \lambda/2$. \\
We can find the phase by considering amplitude:
$$A_1 cos(\pi/3) = A_2 cos(\phi) $$
Which gives 
$$ \phi = tan^{-1} \left(\frac{1}{2} tan(\pi/3)\right) = 0.7137\ radians $$

We can use this value to find the amplitude.
$$ - A_1 \ sin\left(\frac{\pi}{3} \right) = - A_2\ 2\ sin(\phi) $$
$$ A_2 = A_1 \ \frac{sin\left(\frac{\pi}{3} \right)}{2\ sin(\phi)} = 0.6614 \cdot A_1 $$

\begin{figure}[h!]
	\centering
	\includegraphics[width=.6\linewidth]{./modern_phys_hw5_boundary.png}
	\caption{Wave behavior at the boundary condition, with $A_1$ = 1.}
\end{figure}

\newpage
\problem
\begin{question}
A particle is described by the wave function:\\
$$\psi(x) = b(a^2 - x^2) for -a \leq x \leq +a$$
$$\psi(x) = 0\ for\ x \geq -a\ and\ x \geq +a$$
where a and b are positive real constants.\\
(a) Using the normalization condition, find b in terms of a.\\
(b) What is the probability to find the particle at x = +a/2 in a small interval of width 0.010a? \\
(c) What is the probability for the particle to be found between x = +a/2 and x = +a? 
\end{question}

A)\\
Because the wave function is valued zero outside the interval $\left[-a, a\right]$, we can set these as the integration bounds.
$$ \int_{-a}^a P(x) dx = \int_{-a}^a |\psi(x)|^2 dx = 1 $$
Because $\psi(x)$ is a real-valued function, the complex conjugate simplifies to the square.
$$ \int_{-a}^a (b(a^2 - x^2))^2 dx = 1 $$
$$ b^2 \int_{-a}^a (a^2 - x^2)^2 dx = 1 $$
$$ b^2 \int_{-a}^a (a^4 - 2x^2a^2 + x^4)\ dx = 1 $$
$$ b^2 \left[a^4x - (2/3)x^3a^2 + x^5/5 \right]^a_{-a} = 1 $$
$$ b^2 \left[ \left(a^5 - (2/3)a^5 + a^5/5 \right) - \left(-a^5 + (2/3)a^5 - a^5/5 \right) \right] = 1 $$

I shoulda known it was symmetric, oh well
$$ 2b^2 a^5 \left(1 - (2/3) + 1/5 \right) = 1 $$
$$ b^2 \left(1 - (2/3) + 1/5 \right) = \frac{1}{2a^5} $$
$$ b^2  = \frac{1}{2a^5(15/15 - 10/15 + 3/15)} $$
$$ b  = \sqrt{\frac{15}{16a^5}} $$

B)\\
We can approximate the integral by finding the value at a/2, and multiplying it by 0.01a. 
$$ \frac{15}{16a^5} (a^2 - x^2)^2 dx \ for\ x = a/2\ ,\ dx = 0.01a $$
$$ \frac{15}{16a^5} (a^2 - a^2/4)^2 \cdot 0.01a $$
$$ \frac{15}{16a^5} (3/4\ a^2)^2 \cdot 0.01a $$
$$ \frac{15}{16a^5} 9/16\ a^4 \cdot 0.01a $$
$$ \frac{15}{16a^5} \frac{9}{1600}\ a^5 $$
the $a^5$'s cancel out
$$ \frac{15}{16} \frac{9}{1600} $$
0.00527 is the probability of finding the particle at a small slice 0.01a at 1/2a.\\

\newpage
C)\\
$$ \frac{15}{16a^5} \int_{a/2}^a (a^2 - x^2)^2 dx = $$
$$ \frac{15}{16a^5} \left[a^4x - 2/3 x^3a^2 + 1/5 x^5 \right]^a_{a/2} = $$
$$ \frac{15}{16a^5} \left[(a^5 - 2/3 a^5 + 1/5 a^5) - (a^5/2 - 2/3 \cdot 1/8 \cdot a^5 + 1/5 \cdot 1/32 \cdot a^5) \right] = $$
$$ \frac{15}{16} \left[(15/15 - 10/15 + 3/15) - (1/2 - 2/24 + 1/160) \right] = $$
$$ 1/2 - \frac{15}{16}(0.4229166) = $$
$$ 0.1035 $$




\end{document}
