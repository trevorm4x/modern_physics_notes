\documentclass{article}
\usepackage[margin=0.5in]{geometry}
\usepackage{titlesec}
\usepackage{graphicx}
\usepackage{enumitem}
\usepackage{ifthen}
\usepackage{fancyhdr}
\usepackage{xcolor}

% -------- %
% SECTIONS %
% -------- %
\newcounter{problemnumber}\setcounter{problemnumber}{1}
\titlespacing\section{0pt}{10pt}{0pt}   % Spacing between Problems
\titlespacing\subsection{0pt}{5pt}{0pt} % Spacing between Parts
\newcommand{\problem}[1][-1]{
    \setcounter{partnumber}{1}
    \ifnum#1>0
        \setcounter{problemnumber}{#1}
    \fi
    \section*{Problem \arabic{problemnumber}}
    \stepcounter{problemnumber}
}

\newcounter{partnumber}\setcounter{partnumber}{1}
\newcommand{\ppart}[1][-1]{
    \ifnum#1>0
        \setcounter{partnumber}{#1}
    \fi
    \subsection*{Part \parttype{partnumber}}
    \stepcounter{partnumber}
}

\newenvironment{question}{
    \color{gray}\itshape
    \vspace{5pt}
    \begin{tabular}{|p{0.97\linewidth}}
}{
    \end{tabular}\\[5pt]
}



% ------------- %
% HEADER/FOOTER %
% ------------- %
\setlength\parindent{0pt}
\setlength\headheight{30pt}
\headsep=0.25in
\pagestyle{fancy}
\lhead{\ifthenelse{\thepage=1}
    {\textbf{Trevor Smith} \\ \textbf{\writeday}}
}
\chead{\ifthenelse{\thepage=1}
    {\textbf{\huge{HOMEWORK \hwnumber}}}
    {\textbf{\large{HOMEWORK \hwnumber}}}
}
\rhead{\ifthenelse{\thepage=1}
    {\textbf{{\course}} \\ \textbf{Professor {\prof}}}
}
\cfoot{\thepage}
\renewcommand\headrulewidth{0.4pt}
\renewcommand\footrulewidth{0.4pt}



% ---------- %
% PARAMETERS %
% ---------- %
% \PARTTYPE:
% \Alph   := "Part A, Part B,  ..."
% \alph   := "Part a, Part b,  ..."
% \arabic := "Part 1, Part 2,  ..."
% \Roman  := "Part I, Part II, ..."
\newcommand\parttype{\Roman}

% \COURSE:
\newcommand\course{PHYS 2303}

% \HWNUMBER
\newcommand\hwnumber{8}

% \SEMESTER
\newcommand\semester{Spring 2021}

% \PROF
\newcommand\prof{Skinnari}

% \WRITEDAY
% \today is date of compilation, replace if writing due date rather than write date
\newcommand\writeday{\today}



%  ------- %
% DOCUMENT %
% -------- %
\begin{document}

\problem
\begin{question}
	(a) How many different sets of quantum numbers ($n, l, m_l, m_s$) are possible for an electron in the 4f level of an atom? Explain your answer. \\
(b) Suppose that an atom has four electrons in the 4f level. What is the maximum
possible value for the total $m_s$ of the four electrons? \\
(c) What is the maximum possible total $m_l$ of four 4f electrons? \\
(d) Suppose an atom instead has ten electrons in the 4f level. What is the maximum possible value for the total $m_s$ of the ten 4f electrons? \\
(e) What is the maximum possible total $m_l$ of the ten 4f electrons?
\end{question}

\begin{enumerate}[label=\textbf{\alph*.)}]
	\item The n value is set. For n=4, 4 values of l are possible. For each value of l, l*2+1 values of $m_l$ are possible. And for all those, we multiply by two for the two possible values of spin. Thinking only of l and $m_l$, \\
		\begin{table}[h]
		\begin{tabular}{l|l|l}
			$l$ & $m_l$ & total \\\hline
			0   &  0  & 1 \\
			1   &  -1, 0, 1  & 3 \\
			2   &  -2, -1, 0, 1, 2  & 5 \\
			3   &  -3, -2, -1, 0, 1, 2, 3  & 7 \\
		\end{tabular}
		\end{table}
		There are 16 possible combinations of l and $m_l$. We can multiply this by two for spin and get a total of 32.
	\item  If all the electrons are +1/2, the total will be 2.
	\item  The highest $m_l$ is 3, which two electrons can have. 3 + 3 + 2 + 2 = 10.
	\item  After 7 electrons fill the available + spin spots, they will have to be negative. So the total is the same, 2.
	\item  We'll have to fill all the l=3 $m_l$s before hitting the higher ones a second time for a higher value. So 7 go 1 each, then 3 give 3,2,1 for a total of 6.
\end{enumerate}

\problem
\begin{question}
(a) What is the electronic configuration of Fe (Z=26)?
(b) In its ground state, what is the maximum possible total $m_s$ of its electrons?
(c) When the electrons have their maximum possible total $m_s$ , what is the maximum
total $m_l$ ?
(d) Suppose one of the d electrons is excited to the next highest level. What is the
maximum possible total $m_s$ , and when $m_s$ has its maximum total what is the maximum
total $m_l$ ? 
\end{question}

\begin{enumerate}[label=\textbf{\alph*.)}]
	\item $[Ar]4s^23d^6$
	\item  All the full groups will have a total of zero. There must be one pair of electrons in the 3d layer, so there will be 4 upaired ones. If they're all positive, that gives +2.
	\item If the +2 l is the full one, we get +2.
	\item We get the maximum total $m_s$ if the paired electron is the one that's excited, adding 1 for +3. In this situation all the 3d electrons are unpaired which means the total $m_l$ for them is zero. To get the highest possible $m_l$ the excited electron must be in the $4s^1$ +2. 
\end{enumerate}

\newpage
\problem
\begin{question}
The ground state of singly ionized lithium (Z = 3) is 1s². Use the electron screening
model to predict the energies of the 1s¹2p¹ and 1s¹3d¹ excited states for singly ionized
lithium?
Compare your predictions with the measured energies (−13.4 eV and −6.0 eV,
respectively).
\end{question}
\textbf{$1s^12p^1$:}\\
The excited electron is screened by the one remaining 1s electron. 
$$ Z_{eff} = 3-1 = 2 $$
$$ E_n = \left(-13.6\right) \frac{Z_{eff}^2}{n^2} $$
$$ E_n = \left(-13.6\right) \frac{2^2}{2^2} = -13.6 eV $$
There is pretty good agreement with the measured value, implying there is some penetration of the 2p electron into the 1s layer. 

\textbf{$1s^12d^1$:}\\
$$ E_n = \left(-13.6\right) \frac{2^2}{3^2} = -6.0 eV $$
There is exact agreement here, implying that there is no penetration of the exited electron into the screened layer.

\problem
\begin{question}
The molecular vibrational energy of CO is 0.2691eV, when the constituents of the
molecule are the most abundant isotopes of carbon (m = 12.00 u) and oxygen (m =
16.00 u).
(a) What would be the vibrational energy if the oxygen were replaced by the less
abundant isotope with mass 18.00 u?
(b) What would be the vibrational energy if the carbon in the original CO were instead
replaced with radioactive carbon (used in radiocarbon dating) with mass 14.00 u?
\end{question}

\end{document}
